\documentclass[a4paper,12pt]{article}


\usepackage{amssymb,amsmath,array}
\usepackage{hyperref}
\usepackage{bm}

\usepackage[a4paper, left=3cm, right=2.5cm, top=3cm, bottom=3cm]{geometry}
% language and encoding
\usepackage[utf8]{inputenc}
\usepackage[T1]{fontenc}
\usepackage[english]{babel}
% this package load language specific quotes and enables you to write
% \enquote{text} instead of "`text"' or something like that.
\usepackage{csquotes}
\newcommand{\ts}{\textsuperscript}

% Write initials inside the right margin
\newcommand{\initials}[1]{\marginpar{\quad\texttt{#1}}}

\title{Report of the 6\ts{th} exercise sheet}
\author{Marco Adamczyk (MA), Till Brinkmann (TB)}

% could removed
\date{Submission by 17\ts{th} January 2020}


\begin{document}

\pagenumbering{gobble}

\pagestyle{myheadings}
\markright{Till Brinkmann, Marco Adamczyk}
    
\maketitle

\begin{center}
    \textbf{Tutorial: Tuesday 10-12, Tutor: Riza Velioglu}
\end{center}

\section{Introduction}
\subsection{Dataset}
The dataset consists of 896 pictures in 4 different classes. The pictures are 299 by 299 RGB-pixels big.\\
All pictures are nature photographs and seem to be classified as 
"forest" (class 0, 230 pictures), "stony mountains" (class 1, 412 pictures), "snowy mountains" (class 2, 126 pictures), "lake" (class 3, 128 pictures).
Because class 1 has a lot more elements than the other, it could be necessary to balance the dataset. 
\initials{}
\subsection{Initial Features}
TODO Features we tried first
\initials{}
\subsection{Methods/Models}
TODO Which classifiers are used? How do they classify/differ? What are the hyperparameters?
\initials{}

\section{Experiments}
TODO Which tests were done in the experiments? What was implemented? What measurement are used in the results?
Which values were chosen for the hyperparameters?
\initials{}
\subsection{Results}
TODO
\initials{}

\section{Discussion}
TODO Short summary and future work.
\initials{}
\subsection{Additional Features}
TODO
\initials{}

\end{document}