\documentclass[a4paper,12pt]{article}


\usepackage{amssymb,amsmath,array}
\usepackage{hyperref}
\usepackage{bm}

\usepackage[a4paper, left=3cm, right=2.5cm, top=3cm, bottom=3cm]{geometry}
% language and encoding
\usepackage[utf8]{inputenc}
\usepackage[T1]{fontenc}
\usepackage[english]{babel}
% this package load language specific quotes and enables you to write
% \enquote{text} instead of "`text"' or something like that.
\usepackage{csquotes}
\newcommand{\ts}{\textsuperscript}

% Write initials inside the right margin
\newcommand{\initials}[1]{\marginpar{\quad\texttt{#1}}}

\title{Report of the 6\ts{th} exercise sheet}
\author{First group member (FGM), Second group member (SGM), \and
Third group member (TGM)}

% could removed
\date{Submission by 17\ts{th} January 2020}


\begin{document}

\pagenumbering{gobble}

\pagestyle{myheadings}
\markright{First group member, Second group member, Third group member}
    
\maketitle

\begin{center}
    \textbf{Tutorial: Monday 12-14, Tutor: tutor's name}
\end{center}

\section{Introduction}
This is an template for a report. You may use the outline or change it freely. There are no directives.


\section{Methods/Models}
Which classifiers are used? How do they classify/differ?
\initials{FGM}

\section{Experiments}
Which tests were done in the experiments? What was implemented? What measurement are used in the results?
\subsection{Data}
Which data are used? What are their characteristics?
\initials{SGM}

\subsection{Feature Extraction}
Which features are generated for the experiments?
\initials{TGM}

\subsection{Awesome results}

\begin{figure}[h]
\centering
%\includegraphics[width=2.5in]{myfigure}
% where an .eps filename suffix will be assumed under latex, 
% and a .pdf suffix will be assumed for pdflatex; or what has been declared
% via \DeclareGraphicsExtensions.
\caption{Results}
\label{fig_res}
\end{figure}


You can refer to Figure~\ref{fig_res}.
\begin{table}[h]
\caption{An Example of a Table}
\label{tab_example}
\centering
% Some packages, such as MDW tools, offer better commands for making tables
% than the plain LaTeX2e tabular which is used here.
\begin{tabular}{c||c|c|c}
Data & Method 1 & Method 2 & Method 3\\
\hline\hline
data 1 &0.54 & 0.6& 0.98\\
\hline
data 2 &0.74 & 0.54& 0.48\\
\hline
data 3 &0.82 & 0.71& 0.67
\end{tabular}
\end{table}
You can also refer to Table~\ref{tab_example}.
\initials{FGM}

\section{Discussion}
Short summery and future work.
\initials{SGM/TGM}

\end{document}